\documentclass[]{article}

\begin{document}

\title{Project Narrative: Improving Modeling of Process Heterogeneity on Fossil-Based Phylogenetics}
\author{Wright}
\date{Today}
\maketitle

\textbf{Project Summary}

Phylogenetic trees have become crucial components of most comparative biological studies. 
Phylogenetic trees display the relationships between taxa of interest, and if scaled to absolute time, can tell a reader how long different taxa have been evolving independently of one another. 
This information is often an end goal, but phylogentic trees are often also used in further analyses.
For example, the field of phylogenetic comparative methods use the information encoded in a phylogeny (the relationships and timing of divergences between groups) to make inferences about the evolution of traits and organisms.\par
The most important source of information to scale phylogenetic trees to absolute time is the fossil record. 
However, most phylogenetic trees are built with molecular data, such as nucleotide sequence data, and the fossil record rarely yields such information.
Most fossils are known from codifying morphological data.
Comparatively little work has been done to unite morphology and molecular data for divergence time estimation. \par
I will expand the toolkit available to work with both molecular and morphological data in the same time-scaling analysis. 



\end{document}