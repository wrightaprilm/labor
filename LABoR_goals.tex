\documentclass[]{article}

\begin{document}

\title{Project Narrative: Improving Modeling of Process Heterogeneity on Fossil-Based Phylogenetics}
\author{Wright}
\date{Today}
\maketitle

\textbf{Goals}
Below is a statement of the goals of the proposed project. All work will be completed using molecular and morphological data from the ant (Formicidae) group. Ants have a rich fossil record, including many precisely dated specimens.
Because of their interest to agriculture and ecology, there are also many genetic resources available for ants.
Together, these factors make ants an ideal system for studying, understanding, and developing divergence time estimation methods:

\begin{itemize}
\item My first goal is to examine adding realism to the modeling of ant divergence dates by allowing for the parameters of the models for divergence time estimation to vary through time.
Because of their rich fossil record, there are periods in which many ant fossils exist, and periods in which there are few.
This suggests that the evolutionary processes that lead to the formation of ant lineages vary through time.
Therefore, I am interested in exploring models that allow for variation of model parameters over time.
Particularly, I am interested in comparing time-stratified models for divergence dating and models that do not allow for such variation.
\item My second goal is to explore if allowing model parameters to vary across the tree would be a better fit to the ant data.
A second possible answer to why there are periods of extreme ant diversity in the fossil record is that perhaps certain groups of ants fossilize better.
For example, some ants are arboreal, and become preserved in tree sap.
Time-stratified models are simpler, and paleontological datasets are often strongly limited in terms of how much more data can be collected.
In this goal, I will compare the fit of time-stratified models to among-group variation models for divergence time estimation. 
\item \textbf{Final Goal} My final goal is to use the results of goals one and two to build a time-scaled phylogeny of ants. 
With 666 taxa with molecular data, and 158 taxa with morphological data, this will be the largest ant tree to date.
\par

\end{itemize}


\par


\end{document}